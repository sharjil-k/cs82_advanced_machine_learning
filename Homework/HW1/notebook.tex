
% Default to the notebook output style

    


% Inherit from the specified cell style.




    
\documentclass[11pt]{article}

    
    
    \usepackage[T1]{fontenc}
    % Nicer default font (+ math font) than Computer Modern for most use cases
    \usepackage{mathpazo}

    % Basic figure setup, for now with no caption control since it's done
    % automatically by Pandoc (which extracts ![](path) syntax from Markdown).
    \usepackage{graphicx}
    % We will generate all images so they have a width \maxwidth. This means
    % that they will get their normal width if they fit onto the page, but
    % are scaled down if they would overflow the margins.
    \makeatletter
    \def\maxwidth{\ifdim\Gin@nat@width>\linewidth\linewidth
    \else\Gin@nat@width\fi}
    \makeatother
    \let\Oldincludegraphics\includegraphics
    % Set max figure width to be 80% of text width, for now hardcoded.
    \renewcommand{\includegraphics}[1]{\Oldincludegraphics[width=.8\maxwidth]{#1}}
    % Ensure that by default, figures have no caption (until we provide a
    % proper Figure object with a Caption API and a way to capture that
    % in the conversion process - todo).
    \usepackage{caption}
    \DeclareCaptionLabelFormat{nolabel}{}
    \captionsetup{labelformat=nolabel}

    \usepackage{adjustbox} % Used to constrain images to a maximum size 
    \usepackage{xcolor} % Allow colors to be defined
    \usepackage{enumerate} % Needed for markdown enumerations to work
    \usepackage{geometry} % Used to adjust the document margins
    \usepackage{amsmath} % Equations
    \usepackage{amssymb} % Equations
    \usepackage{textcomp} % defines textquotesingle
    % Hack from http://tex.stackexchange.com/a/47451/13684:
    \AtBeginDocument{%
        \def\PYZsq{\textquotesingle}% Upright quotes in Pygmentized code
    }
    \usepackage{upquote} % Upright quotes for verbatim code
    \usepackage{eurosym} % defines \euro
    \usepackage[mathletters]{ucs} % Extended unicode (utf-8) support
    \usepackage[utf8x]{inputenc} % Allow utf-8 characters in the tex document
    \usepackage{fancyvrb} % verbatim replacement that allows latex
    \usepackage{grffile} % extends the file name processing of package graphics 
                         % to support a larger range 
    % The hyperref package gives us a pdf with properly built
    % internal navigation ('pdf bookmarks' for the table of contents,
    % internal cross-reference links, web links for URLs, etc.)
    \usepackage{hyperref}
    \usepackage{longtable} % longtable support required by pandoc >1.10
    \usepackage{booktabs}  % table support for pandoc > 1.12.2
    \usepackage[inline]{enumitem} % IRkernel/repr support (it uses the enumerate* environment)
    \usepackage[normalem]{ulem} % ulem is needed to support strikethroughs (\sout)
                                % normalem makes italics be italics, not underlines
    

    
    
    % Colors for the hyperref package
    \definecolor{urlcolor}{rgb}{0,.145,.698}
    \definecolor{linkcolor}{rgb}{.71,0.21,0.01}
    \definecolor{citecolor}{rgb}{.12,.54,.11}

    % ANSI colors
    \definecolor{ansi-black}{HTML}{3E424D}
    \definecolor{ansi-black-intense}{HTML}{282C36}
    \definecolor{ansi-red}{HTML}{E75C58}
    \definecolor{ansi-red-intense}{HTML}{B22B31}
    \definecolor{ansi-green}{HTML}{00A250}
    \definecolor{ansi-green-intense}{HTML}{007427}
    \definecolor{ansi-yellow}{HTML}{DDB62B}
    \definecolor{ansi-yellow-intense}{HTML}{B27D12}
    \definecolor{ansi-blue}{HTML}{208FFB}
    \definecolor{ansi-blue-intense}{HTML}{0065CA}
    \definecolor{ansi-magenta}{HTML}{D160C4}
    \definecolor{ansi-magenta-intense}{HTML}{A03196}
    \definecolor{ansi-cyan}{HTML}{60C6C8}
    \definecolor{ansi-cyan-intense}{HTML}{258F8F}
    \definecolor{ansi-white}{HTML}{C5C1B4}
    \definecolor{ansi-white-intense}{HTML}{A1A6B2}

    % commands and environments needed by pandoc snippets
    % extracted from the output of `pandoc -s`
    \providecommand{\tightlist}{%
      \setlength{\itemsep}{0pt}\setlength{\parskip}{0pt}}
    \DefineVerbatimEnvironment{Highlighting}{Verbatim}{commandchars=\\\{\}}
    % Add ',fontsize=\small' for more characters per line
    \newenvironment{Shaded}{}{}
    \newcommand{\KeywordTok}[1]{\textcolor[rgb]{0.00,0.44,0.13}{\textbf{{#1}}}}
    \newcommand{\DataTypeTok}[1]{\textcolor[rgb]{0.56,0.13,0.00}{{#1}}}
    \newcommand{\DecValTok}[1]{\textcolor[rgb]{0.25,0.63,0.44}{{#1}}}
    \newcommand{\BaseNTok}[1]{\textcolor[rgb]{0.25,0.63,0.44}{{#1}}}
    \newcommand{\FloatTok}[1]{\textcolor[rgb]{0.25,0.63,0.44}{{#1}}}
    \newcommand{\CharTok}[1]{\textcolor[rgb]{0.25,0.44,0.63}{{#1}}}
    \newcommand{\StringTok}[1]{\textcolor[rgb]{0.25,0.44,0.63}{{#1}}}
    \newcommand{\CommentTok}[1]{\textcolor[rgb]{0.38,0.63,0.69}{\textit{{#1}}}}
    \newcommand{\OtherTok}[1]{\textcolor[rgb]{0.00,0.44,0.13}{{#1}}}
    \newcommand{\AlertTok}[1]{\textcolor[rgb]{1.00,0.00,0.00}{\textbf{{#1}}}}
    \newcommand{\FunctionTok}[1]{\textcolor[rgb]{0.02,0.16,0.49}{{#1}}}
    \newcommand{\RegionMarkerTok}[1]{{#1}}
    \newcommand{\ErrorTok}[1]{\textcolor[rgb]{1.00,0.00,0.00}{\textbf{{#1}}}}
    \newcommand{\NormalTok}[1]{{#1}}
    
    % Additional commands for more recent versions of Pandoc
    \newcommand{\ConstantTok}[1]{\textcolor[rgb]{0.53,0.00,0.00}{{#1}}}
    \newcommand{\SpecialCharTok}[1]{\textcolor[rgb]{0.25,0.44,0.63}{{#1}}}
    \newcommand{\VerbatimStringTok}[1]{\textcolor[rgb]{0.25,0.44,0.63}{{#1}}}
    \newcommand{\SpecialStringTok}[1]{\textcolor[rgb]{0.73,0.40,0.53}{{#1}}}
    \newcommand{\ImportTok}[1]{{#1}}
    \newcommand{\DocumentationTok}[1]{\textcolor[rgb]{0.73,0.13,0.13}{\textit{{#1}}}}
    \newcommand{\AnnotationTok}[1]{\textcolor[rgb]{0.38,0.63,0.69}{\textbf{\textit{{#1}}}}}
    \newcommand{\CommentVarTok}[1]{\textcolor[rgb]{0.38,0.63,0.69}{\textbf{\textit{{#1}}}}}
    \newcommand{\VariableTok}[1]{\textcolor[rgb]{0.10,0.09,0.49}{{#1}}}
    \newcommand{\ControlFlowTok}[1]{\textcolor[rgb]{0.00,0.44,0.13}{\textbf{{#1}}}}
    \newcommand{\OperatorTok}[1]{\textcolor[rgb]{0.40,0.40,0.40}{{#1}}}
    \newcommand{\BuiltInTok}[1]{{#1}}
    \newcommand{\ExtensionTok}[1]{{#1}}
    \newcommand{\PreprocessorTok}[1]{\textcolor[rgb]{0.74,0.48,0.00}{{#1}}}
    \newcommand{\AttributeTok}[1]{\textcolor[rgb]{0.49,0.56,0.16}{{#1}}}
    \newcommand{\InformationTok}[1]{\textcolor[rgb]{0.38,0.63,0.69}{\textbf{\textit{{#1}}}}}
    \newcommand{\WarningTok}[1]{\textcolor[rgb]{0.38,0.63,0.69}{\textbf{\textit{{#1}}}}}
    
    
    % Define a nice break command that doesn't care if a line doesn't already
    % exist.
    \def\br{\hspace*{\fill} \\* }
    % Math Jax compatability definitions
    \def\gt{>}
    \def\lt{<}
    % Document parameters
    \title{KhanSharjilHW1}
    
    
    

    % Pygments definitions
    
\makeatletter
\def\PY@reset{\let\PY@it=\relax \let\PY@bf=\relax%
    \let\PY@ul=\relax \let\PY@tc=\relax%
    \let\PY@bc=\relax \let\PY@ff=\relax}
\def\PY@tok#1{\csname PY@tok@#1\endcsname}
\def\PY@toks#1+{\ifx\relax#1\empty\else%
    \PY@tok{#1}\expandafter\PY@toks\fi}
\def\PY@do#1{\PY@bc{\PY@tc{\PY@ul{%
    \PY@it{\PY@bf{\PY@ff{#1}}}}}}}
\def\PY#1#2{\PY@reset\PY@toks#1+\relax+\PY@do{#2}}

\expandafter\def\csname PY@tok@w\endcsname{\def\PY@tc##1{\textcolor[rgb]{0.73,0.73,0.73}{##1}}}
\expandafter\def\csname PY@tok@c\endcsname{\let\PY@it=\textit\def\PY@tc##1{\textcolor[rgb]{0.25,0.50,0.50}{##1}}}
\expandafter\def\csname PY@tok@cp\endcsname{\def\PY@tc##1{\textcolor[rgb]{0.74,0.48,0.00}{##1}}}
\expandafter\def\csname PY@tok@k\endcsname{\let\PY@bf=\textbf\def\PY@tc##1{\textcolor[rgb]{0.00,0.50,0.00}{##1}}}
\expandafter\def\csname PY@tok@kp\endcsname{\def\PY@tc##1{\textcolor[rgb]{0.00,0.50,0.00}{##1}}}
\expandafter\def\csname PY@tok@kt\endcsname{\def\PY@tc##1{\textcolor[rgb]{0.69,0.00,0.25}{##1}}}
\expandafter\def\csname PY@tok@o\endcsname{\def\PY@tc##1{\textcolor[rgb]{0.40,0.40,0.40}{##1}}}
\expandafter\def\csname PY@tok@ow\endcsname{\let\PY@bf=\textbf\def\PY@tc##1{\textcolor[rgb]{0.67,0.13,1.00}{##1}}}
\expandafter\def\csname PY@tok@nb\endcsname{\def\PY@tc##1{\textcolor[rgb]{0.00,0.50,0.00}{##1}}}
\expandafter\def\csname PY@tok@nf\endcsname{\def\PY@tc##1{\textcolor[rgb]{0.00,0.00,1.00}{##1}}}
\expandafter\def\csname PY@tok@nc\endcsname{\let\PY@bf=\textbf\def\PY@tc##1{\textcolor[rgb]{0.00,0.00,1.00}{##1}}}
\expandafter\def\csname PY@tok@nn\endcsname{\let\PY@bf=\textbf\def\PY@tc##1{\textcolor[rgb]{0.00,0.00,1.00}{##1}}}
\expandafter\def\csname PY@tok@ne\endcsname{\let\PY@bf=\textbf\def\PY@tc##1{\textcolor[rgb]{0.82,0.25,0.23}{##1}}}
\expandafter\def\csname PY@tok@nv\endcsname{\def\PY@tc##1{\textcolor[rgb]{0.10,0.09,0.49}{##1}}}
\expandafter\def\csname PY@tok@no\endcsname{\def\PY@tc##1{\textcolor[rgb]{0.53,0.00,0.00}{##1}}}
\expandafter\def\csname PY@tok@nl\endcsname{\def\PY@tc##1{\textcolor[rgb]{0.63,0.63,0.00}{##1}}}
\expandafter\def\csname PY@tok@ni\endcsname{\let\PY@bf=\textbf\def\PY@tc##1{\textcolor[rgb]{0.60,0.60,0.60}{##1}}}
\expandafter\def\csname PY@tok@na\endcsname{\def\PY@tc##1{\textcolor[rgb]{0.49,0.56,0.16}{##1}}}
\expandafter\def\csname PY@tok@nt\endcsname{\let\PY@bf=\textbf\def\PY@tc##1{\textcolor[rgb]{0.00,0.50,0.00}{##1}}}
\expandafter\def\csname PY@tok@nd\endcsname{\def\PY@tc##1{\textcolor[rgb]{0.67,0.13,1.00}{##1}}}
\expandafter\def\csname PY@tok@s\endcsname{\def\PY@tc##1{\textcolor[rgb]{0.73,0.13,0.13}{##1}}}
\expandafter\def\csname PY@tok@sd\endcsname{\let\PY@it=\textit\def\PY@tc##1{\textcolor[rgb]{0.73,0.13,0.13}{##1}}}
\expandafter\def\csname PY@tok@si\endcsname{\let\PY@bf=\textbf\def\PY@tc##1{\textcolor[rgb]{0.73,0.40,0.53}{##1}}}
\expandafter\def\csname PY@tok@se\endcsname{\let\PY@bf=\textbf\def\PY@tc##1{\textcolor[rgb]{0.73,0.40,0.13}{##1}}}
\expandafter\def\csname PY@tok@sr\endcsname{\def\PY@tc##1{\textcolor[rgb]{0.73,0.40,0.53}{##1}}}
\expandafter\def\csname PY@tok@ss\endcsname{\def\PY@tc##1{\textcolor[rgb]{0.10,0.09,0.49}{##1}}}
\expandafter\def\csname PY@tok@sx\endcsname{\def\PY@tc##1{\textcolor[rgb]{0.00,0.50,0.00}{##1}}}
\expandafter\def\csname PY@tok@m\endcsname{\def\PY@tc##1{\textcolor[rgb]{0.40,0.40,0.40}{##1}}}
\expandafter\def\csname PY@tok@gh\endcsname{\let\PY@bf=\textbf\def\PY@tc##1{\textcolor[rgb]{0.00,0.00,0.50}{##1}}}
\expandafter\def\csname PY@tok@gu\endcsname{\let\PY@bf=\textbf\def\PY@tc##1{\textcolor[rgb]{0.50,0.00,0.50}{##1}}}
\expandafter\def\csname PY@tok@gd\endcsname{\def\PY@tc##1{\textcolor[rgb]{0.63,0.00,0.00}{##1}}}
\expandafter\def\csname PY@tok@gi\endcsname{\def\PY@tc##1{\textcolor[rgb]{0.00,0.63,0.00}{##1}}}
\expandafter\def\csname PY@tok@gr\endcsname{\def\PY@tc##1{\textcolor[rgb]{1.00,0.00,0.00}{##1}}}
\expandafter\def\csname PY@tok@ge\endcsname{\let\PY@it=\textit}
\expandafter\def\csname PY@tok@gs\endcsname{\let\PY@bf=\textbf}
\expandafter\def\csname PY@tok@gp\endcsname{\let\PY@bf=\textbf\def\PY@tc##1{\textcolor[rgb]{0.00,0.00,0.50}{##1}}}
\expandafter\def\csname PY@tok@go\endcsname{\def\PY@tc##1{\textcolor[rgb]{0.53,0.53,0.53}{##1}}}
\expandafter\def\csname PY@tok@gt\endcsname{\def\PY@tc##1{\textcolor[rgb]{0.00,0.27,0.87}{##1}}}
\expandafter\def\csname PY@tok@err\endcsname{\def\PY@bc##1{\setlength{\fboxsep}{0pt}\fcolorbox[rgb]{1.00,0.00,0.00}{1,1,1}{\strut ##1}}}
\expandafter\def\csname PY@tok@kc\endcsname{\let\PY@bf=\textbf\def\PY@tc##1{\textcolor[rgb]{0.00,0.50,0.00}{##1}}}
\expandafter\def\csname PY@tok@kd\endcsname{\let\PY@bf=\textbf\def\PY@tc##1{\textcolor[rgb]{0.00,0.50,0.00}{##1}}}
\expandafter\def\csname PY@tok@kn\endcsname{\let\PY@bf=\textbf\def\PY@tc##1{\textcolor[rgb]{0.00,0.50,0.00}{##1}}}
\expandafter\def\csname PY@tok@kr\endcsname{\let\PY@bf=\textbf\def\PY@tc##1{\textcolor[rgb]{0.00,0.50,0.00}{##1}}}
\expandafter\def\csname PY@tok@bp\endcsname{\def\PY@tc##1{\textcolor[rgb]{0.00,0.50,0.00}{##1}}}
\expandafter\def\csname PY@tok@fm\endcsname{\def\PY@tc##1{\textcolor[rgb]{0.00,0.00,1.00}{##1}}}
\expandafter\def\csname PY@tok@vc\endcsname{\def\PY@tc##1{\textcolor[rgb]{0.10,0.09,0.49}{##1}}}
\expandafter\def\csname PY@tok@vg\endcsname{\def\PY@tc##1{\textcolor[rgb]{0.10,0.09,0.49}{##1}}}
\expandafter\def\csname PY@tok@vi\endcsname{\def\PY@tc##1{\textcolor[rgb]{0.10,0.09,0.49}{##1}}}
\expandafter\def\csname PY@tok@vm\endcsname{\def\PY@tc##1{\textcolor[rgb]{0.10,0.09,0.49}{##1}}}
\expandafter\def\csname PY@tok@sa\endcsname{\def\PY@tc##1{\textcolor[rgb]{0.73,0.13,0.13}{##1}}}
\expandafter\def\csname PY@tok@sb\endcsname{\def\PY@tc##1{\textcolor[rgb]{0.73,0.13,0.13}{##1}}}
\expandafter\def\csname PY@tok@sc\endcsname{\def\PY@tc##1{\textcolor[rgb]{0.73,0.13,0.13}{##1}}}
\expandafter\def\csname PY@tok@dl\endcsname{\def\PY@tc##1{\textcolor[rgb]{0.73,0.13,0.13}{##1}}}
\expandafter\def\csname PY@tok@s2\endcsname{\def\PY@tc##1{\textcolor[rgb]{0.73,0.13,0.13}{##1}}}
\expandafter\def\csname PY@tok@sh\endcsname{\def\PY@tc##1{\textcolor[rgb]{0.73,0.13,0.13}{##1}}}
\expandafter\def\csname PY@tok@s1\endcsname{\def\PY@tc##1{\textcolor[rgb]{0.73,0.13,0.13}{##1}}}
\expandafter\def\csname PY@tok@mb\endcsname{\def\PY@tc##1{\textcolor[rgb]{0.40,0.40,0.40}{##1}}}
\expandafter\def\csname PY@tok@mf\endcsname{\def\PY@tc##1{\textcolor[rgb]{0.40,0.40,0.40}{##1}}}
\expandafter\def\csname PY@tok@mh\endcsname{\def\PY@tc##1{\textcolor[rgb]{0.40,0.40,0.40}{##1}}}
\expandafter\def\csname PY@tok@mi\endcsname{\def\PY@tc##1{\textcolor[rgb]{0.40,0.40,0.40}{##1}}}
\expandafter\def\csname PY@tok@il\endcsname{\def\PY@tc##1{\textcolor[rgb]{0.40,0.40,0.40}{##1}}}
\expandafter\def\csname PY@tok@mo\endcsname{\def\PY@tc##1{\textcolor[rgb]{0.40,0.40,0.40}{##1}}}
\expandafter\def\csname PY@tok@ch\endcsname{\let\PY@it=\textit\def\PY@tc##1{\textcolor[rgb]{0.25,0.50,0.50}{##1}}}
\expandafter\def\csname PY@tok@cm\endcsname{\let\PY@it=\textit\def\PY@tc##1{\textcolor[rgb]{0.25,0.50,0.50}{##1}}}
\expandafter\def\csname PY@tok@cpf\endcsname{\let\PY@it=\textit\def\PY@tc##1{\textcolor[rgb]{0.25,0.50,0.50}{##1}}}
\expandafter\def\csname PY@tok@c1\endcsname{\let\PY@it=\textit\def\PY@tc##1{\textcolor[rgb]{0.25,0.50,0.50}{##1}}}
\expandafter\def\csname PY@tok@cs\endcsname{\let\PY@it=\textit\def\PY@tc##1{\textcolor[rgb]{0.25,0.50,0.50}{##1}}}

\def\PYZbs{\char`\\}
\def\PYZus{\char`\_}
\def\PYZob{\char`\{}
\def\PYZcb{\char`\}}
\def\PYZca{\char`\^}
\def\PYZam{\char`\&}
\def\PYZlt{\char`\<}
\def\PYZgt{\char`\>}
\def\PYZsh{\char`\#}
\def\PYZpc{\char`\%}
\def\PYZdl{\char`\$}
\def\PYZhy{\char`\-}
\def\PYZsq{\char`\'}
\def\PYZdq{\char`\"}
\def\PYZti{\char`\~}
% for compatibility with earlier versions
\def\PYZat{@}
\def\PYZlb{[}
\def\PYZrb{]}
\makeatother


    % Exact colors from NB
    \definecolor{incolor}{rgb}{0.0, 0.0, 0.5}
    \definecolor{outcolor}{rgb}{0.545, 0.0, 0.0}



    
    % Prevent overflowing lines due to hard-to-break entities
    \sloppy 
    % Setup hyperref package
    \hypersetup{
      breaklinks=true,  % so long urls are correctly broken across lines
      colorlinks=true,
      urlcolor=urlcolor,
      linkcolor=linkcolor,
      citecolor=citecolor,
      }
    % Slightly bigger margins than the latex defaults
    
    \geometry{verbose,tmargin=1in,bmargin=1in,lmargin=1in,rmargin=1in}
    
    

    \begin{document}
    
    
    \maketitle
    
    

    
    \#\#

CSCI E-82

\#\#

HW 1 Dimensionality Reduction

\#\#\#

Due: Sept 17, 2018 11:59pm EST

\paragraph{Note that this is an individual homework to be completed
without collaborations except through
Piazza.}\label{note-that-this-is-an-individual-homework-to-be-completed-without-collaborations-except-through-piazza.}

\paragraph{We encourage you to make progress this weekend since the
second homework will likely come out in a week before this one is
due.}\label{we-encourage-you-to-make-progress-this-weekend-since-the-second-homework-will-likely-come-out-in-a-week-before-this-one-is-due.}

    \subsubsection{Your name:}\label{your-name}

    Sharjil Khan

    \subsubsection{Problem 1 (5 points)}\label{problem-1-5-points}

\[\mathbf{X} = \left[\begin{array}
{rrr}
1 & 2 & 3 \\
4 & 5 & 6 \\
7 & 8 & 9
\end{array}\right]
\]

\[\mathbf{Y} = \left[\begin{array}
{rrr}
1 & 2 & 1 \\
2 & 1 & 2  
\end{array}\right]
\]

Compute XYT. The answer can be computed by hand and written in Markdown
like the above matrices, or computed in python. Either way is
acceptable.

    \begin{Verbatim}[commandchars=\\\{\}]
{\color{incolor}In [{\color{incolor}3}]:} \PY{k+kn}{import} \PY{n+nn}{numpy} \PY{k}{as} \PY{n+nn}{np}
        
        \PY{n}{X} \PY{o}{=} \PY{n}{np}\PY{o}{.}\PY{n}{array}\PY{p}{(}\PY{p}{[}\PY{p}{[}\PY{l+m+mi}{1}\PY{p}{,}\PY{l+m+mi}{2}\PY{p}{,}\PY{l+m+mi}{3}\PY{p}{]}\PY{p}{,}\PY{p}{[}\PY{l+m+mi}{4}\PY{p}{,}\PY{l+m+mi}{5}\PY{p}{,}\PY{l+m+mi}{6}\PY{p}{]}\PY{p}{,}\PY{p}{[}\PY{l+m+mi}{7}\PY{p}{,}\PY{l+m+mi}{8}\PY{p}{,}\PY{l+m+mi}{9}\PY{p}{]}\PY{p}{]}\PY{p}{)}
        \PY{n}{Y} \PY{o}{=} \PY{n}{np}\PY{o}{.}\PY{n}{array}\PY{p}{(}\PY{p}{[}\PY{p}{[}\PY{l+m+mi}{1}\PY{p}{,}\PY{l+m+mi}{2}\PY{p}{,}\PY{l+m+mi}{1}\PY{p}{]}\PY{p}{,}\PY{p}{[}\PY{l+m+mi}{2}\PY{p}{,}\PY{l+m+mi}{1}\PY{p}{,}\PY{l+m+mi}{2}\PY{p}{]}\PY{p}{]}\PY{p}{)}
        
        \PY{n+nb}{print}\PY{p}{(}\PY{l+s+s2}{\PYZdq{}}\PY{l+s+s2}{Answer = }\PY{l+s+se}{\PYZbs{}n}\PY{l+s+s2}{\PYZdq{}}\PY{p}{,} \PY{n}{np}\PY{o}{.}\PY{n}{dot}\PY{p}{(}\PY{n}{X}\PY{p}{,}\PY{n}{Y}\PY{o}{.}\PY{n}{T}\PY{p}{)}\PY{p}{)}
\end{Verbatim}


    \begin{Verbatim}[commandchars=\\\{\}]
Answer = 
 [[ 8 10]
 [20 25]
 [32 40]]

    \end{Verbatim}

    \subsubsection{Problem 2}\label{problem-2}

This problem goes through a combination of python data manipulations as
well as the full math projection using PCA. We have divided the problem
into multiple parts.

    \subsubsection{Problem 2a (5 points)}\label{problem-2a-5-points}

Download and load in the data set from the UCI archive
https://archive.ics.uci.edu/ml/machine-learning-databases/ecoli/. Print
the dimensions and the first few rows to demonstrate a successful load.

    \begin{Verbatim}[commandchars=\\\{\}]
{\color{incolor}In [{\color{incolor}4}]:} \PY{k+kn}{import} \PY{n+nn}{pandas} \PY{k}{as} \PY{n+nn}{pd}
        \PY{n}{url} \PY{o}{=} \PY{l+s+s1}{\PYZsq{}}\PY{l+s+s1}{https://archive.ics.uci.edu/ml/machine\PYZhy{}learning\PYZhy{}databases/ecoli/ecoli.data}\PY{l+s+s1}{\PYZsq{}}
        \PY{n}{ecoli} \PY{o}{=} \PY{n}{pd}\PY{o}{.}\PY{n}{read\PYZus{}csv}\PY{p}{(}\PY{n}{url}\PY{p}{,} \PY{n}{delim\PYZus{}whitespace}\PY{o}{=}\PY{k+kc}{True}\PY{p}{,} \PY{n}{header}\PY{o}{=} \PY{k+kc}{None}\PY{p}{)}
        \PY{n}{rows}\PY{p}{,} \PY{n}{columns} \PY{o}{=} \PY{n}{ecoli}\PY{o}{.}\PY{n}{shape}
        \PY{n+nb}{print}\PY{p}{(}\PY{l+s+s2}{\PYZdq{}}\PY{l+s+s2}{Rows: }\PY{l+s+si}{\PYZpc{}d}\PY{l+s+s2}{, Columns: }\PY{l+s+si}{\PYZpc{}d}\PY{l+s+s2}{\PYZdq{}} \PY{o}{\PYZpc{}}\PY{p}{(}\PY{n}{rows}\PY{p}{,} \PY{n}{columns}\PY{p}{)}\PY{p}{)}
        \PY{n}{ecoli}\PY{o}{.}\PY{n}{head}\PY{p}{(}\PY{p}{)}
\end{Verbatim}


    \begin{Verbatim}[commandchars=\\\{\}]
Rows: 336, Columns: 9

    \end{Verbatim}

\begin{Verbatim}[commandchars=\\\{\}]
{\color{outcolor}Out[{\color{outcolor}4}]:}             0     1     2     3    4     5     6     7   8
        0   AAT\_ECOLI  0.49  0.29  0.48  0.5  0.56  0.24  0.35  cp
        1  ACEA\_ECOLI  0.07  0.40  0.48  0.5  0.54  0.35  0.44  cp
        2  ACEK\_ECOLI  0.56  0.40  0.48  0.5  0.49  0.37  0.46  cp
        3  ACKA\_ECOLI  0.59  0.49  0.48  0.5  0.52  0.45  0.36  cp
        4   ADI\_ECOLI  0.23  0.32  0.48  0.5  0.55  0.25  0.35  cp
\end{Verbatim}
            
    \subsubsection{Problem 2b (10 points)}\label{problem-2b-10-points}

Compute and print the covariance matrix for all columns excluding the
first and last. Rather than use the built-in function, compute this
using python code for practice. The following equation will suffice for
this.

Cov(X, Y) = Σ ( Xi - X ) ( Yi - Y ) / N

    \begin{Verbatim}[commandchars=\\\{\}]
{\color{incolor}In [{\color{incolor}5}]:} \PY{k+kn}{from} \PY{n+nn}{sklearn} \PY{k}{import} \PY{n}{decomposition}\PY{p}{,} \PY{n}{preprocessing}
        
        \PY{c+c1}{\PYZsh{}DROP FIRST AND LAST COLUMN }
        \PY{n}{cols} \PY{o}{=} \PY{p}{[}\PY{l+m+mi}{0}\PY{p}{,}\PY{l+m+mi}{8}\PY{p}{]}
        \PY{n}{ecoli\PYZus{}numeric} \PY{o}{=} \PY{n}{ecoli}\PY{o}{.}\PY{n}{drop}\PY{p}{(}\PY{n}{ecoli}\PY{o}{.}\PY{n}{columns}\PY{p}{[}\PY{n}{cols}\PY{p}{]}\PY{p}{,}\PY{n}{axis}\PY{o}{=}\PY{l+m+mi}{1}\PY{p}{)}
        
        \PY{c+c1}{\PYZsh{}SUBTRACT MEAN TO CENTER THE DATA AROUND 0 ON ALL AXIS}
        \PY{n}{mu} \PY{o}{=} \PY{n}{ecoli\PYZus{}numeric}\PY{o}{.}\PY{n}{mean}\PY{p}{(}\PY{n}{axis}\PY{o}{=}\PY{l+m+mi}{0}\PY{p}{)}
        \PY{n}{ecoli\PYZus{}0} \PY{o}{=} \PY{n}{ecoli\PYZus{}numeric} \PY{o}{\PYZhy{}} \PY{n}{mu}
        
        \PY{c+c1}{\PYZsh{}CALCULATE MANUALLY}
        \PY{n}{means} \PY{o}{=} \PY{n}{ecoli\PYZus{}0}\PY{o}{.}\PY{n}{mean}\PY{p}{(}\PY{n}{axis}\PY{o}{=}\PY{l+m+mi}{0}\PY{p}{)}
        \PY{n}{ecoli0} \PY{o}{=} \PY{n}{ecoli\PYZus{}0} \PY{o}{\PYZhy{}} \PY{n}{means}
        \PY{n}{N}\PY{p}{,} \PY{n}{column\PYZus{}num} \PY{o}{=} \PY{n}{ecoli0}\PY{o}{.}\PY{n}{shape}
        
        \PY{c+c1}{\PYZsh{} FUNCTION TO COMPUTE EACH ROW OF THE COVARIANCE MATRIX}
        \PY{k}{def} \PY{n+nf}{covariance\PYZus{}row} \PY{p}{(}\PY{n}{matrix} \PY{p}{,} \PY{n}{column\PYZus{}values} \PY{p}{)}\PY{p}{:}
            \PY{n}{N}\PY{p}{,} \PY{n}{column\PYZus{}num} \PY{o}{=} \PY{n}{matrix}\PY{o}{.}\PY{n}{shape}
            \PY{n}{row} \PY{o}{=} \PY{p}{[}\PY{p}{]}
            \PY{k}{for} \PY{n}{i} \PY{o+ow}{in} \PY{n+nb}{range}\PY{p}{(}\PY{l+m+mi}{1}\PY{p}{,} \PY{n}{column\PYZus{}num}\PY{o}{+}\PY{l+m+mi}{1}\PY{p}{)}\PY{p}{:}
                \PY{n}{dot\PYZus{}prod} \PY{o}{=} \PY{n}{np}\PY{o}{.}\PY{n}{dot}\PY{p}{(}\PY{n}{matrix}\PY{o}{.}\PY{n}{iloc}\PY{p}{[}\PY{p}{:}\PY{p}{,} \PY{p}{(}\PY{n}{i}\PY{o}{\PYZhy{}}\PY{l+m+mi}{1}\PY{p}{)}\PY{p}{:} \PY{n}{i}\PY{p}{]}\PY{o}{.}\PY{n}{T}\PY{p}{,} \PY{n}{column\PYZus{}values}\PY{p}{)}
                \PY{n}{row}\PY{o}{.}\PY{n}{append}\PY{p}{(}\PY{n}{dot\PYZus{}prod}\PY{p}{[}\PY{l+m+mi}{0}\PY{p}{,}\PY{l+m+mi}{0}\PY{p}{]}\PY{o}{/}\PY{n}{N}\PY{p}{)}
            \PY{k}{return} \PY{n}{row}
        \PY{c+c1}{\PYZsh{} FUNCTION TO COMPUTE THE COVARIANCE MATRIX}
        \PY{k}{def} \PY{n+nf}{covariance\PYZus{}matrix}\PY{p}{(}\PY{n}{dataframe}\PY{p}{)}\PY{p}{:}
            \PY{n}{N}\PY{p}{,} \PY{n}{column\PYZus{}num} \PY{o}{=} \PY{n}{dataframe}\PY{o}{.}\PY{n}{shape}
            \PY{n}{cov} \PY{o}{=} \PY{p}{[}\PY{p}{]}
            \PY{k}{for} \PY{n}{i} \PY{o+ow}{in} \PY{n+nb}{range}\PY{p}{(}\PY{l+m+mi}{0}\PY{p}{,}\PY{n}{column\PYZus{}num}\PY{p}{)}\PY{p}{:}
                \PY{n}{cov}\PY{o}{.}\PY{n}{append}\PY{p}{(}\PY{n}{covariance\PYZus{}row}\PY{p}{(}\PY{n}{dataframe}\PY{p}{,} \PY{n}{dataframe}\PY{o}{.}\PY{n}{iloc}\PY{p}{[}\PY{p}{:}\PY{p}{,}\PY{n}{i}\PY{p}{:}\PY{n}{i}\PY{o}{+}\PY{l+m+mi}{1}\PY{p}{]}\PY{p}{)}\PY{p}{)}
            \PY{k}{return} \PY{n}{cov}
        
        \PY{c+c1}{\PYZsh{}GET THE COVARIANCE MATRIX AND PRINT IT}
        \PY{n}{C} \PY{o}{=} \PY{n}{np}\PY{o}{.}\PY{n}{array}\PY{p}{(}\PY{n}{covariance\PYZus{}matrix}\PY{p}{(}\PY{n}{ecoli0}\PY{p}{)}\PY{p}{)}
        \PY{n+nb}{print}\PY{p}{(}\PY{n}{C}\PY{p}{)}
\end{Verbatim}


    \begin{Verbatim}[commandchars=\\\{\}]
[[ 3.77696393e-02  1.30758929e-02  2.52169785e-03  3.71935232e-04
   5.24106966e-03  1.66205251e-02  6.78989690e-03]
 [ 1.30758929e-02  2.18851190e-02  5.72619048e-04  7.44047619e-05
   1.26220238e-03  5.52916667e-03 -3.71815476e-03]
 [ 2.52169785e-03  5.72619048e-04  7.80810658e-03  7.50779478e-04
   7.57872732e-04  1.82342687e-03 -1.06371173e-03]
 [ 3.71935232e-04  7.44047619e-05  7.50779478e-04  7.41833192e-04
  -1.48853812e-04 -4.49085884e-05 -2.97220451e-04]
 [ 5.24106966e-03  1.26220238e-03  7.57872732e-04 -1.48853812e-04
   1.49312491e-02  7.35713754e-03  6.45596035e-03]
 [ 1.66205251e-02  5.52916667e-03  1.82342687e-03 -4.49085884e-05
   7.35713754e-03  4.64100872e-02  3.64568931e-02]
 [ 6.78989690e-03 -3.71815476e-03 -1.06371173e-03 -2.97220451e-04
   6.45596035e-03  3.64568931e-02  4.37222497e-02]]

    \end{Verbatim}

    \subsubsection{Problem 2c (10 points)}\label{problem-2c-10-points}

Compute the decomposition of the covariance matrix using singular value
decomposition. Using a python function is definitely the way to go here.

    \begin{Verbatim}[commandchars=\\\{\}]
{\color{incolor}In [{\color{incolor}8}]:} \PY{c+c1}{\PYZsh{} FIND THE EIGENVALUES AND EIGENVECTORS FROM THE COVARIANCE MATRIX}
        \PY{n}{eigenvalues}\PY{p}{,} \PY{n}{eigenvectors} \PY{o}{=} \PY{n}{np}\PY{o}{.}\PY{n}{linalg}\PY{o}{.}\PY{n}{eigh}\PY{p}{(}\PY{n}{C}\PY{p}{)}
        
        
        \PY{c+c1}{\PYZsh{}SORT THE EIGENVALUES IN DESCENDING ORDER}
        \PY{n}{idx} \PY{o}{=} \PY{n}{np}\PY{o}{.}\PY{n}{argsort}\PY{p}{(}\PY{o}{\PYZhy{}}\PY{n}{eigenvalues}\PY{p}{)}
        \PY{n}{eigenvalues} \PY{o}{=} \PY{n}{eigenvalues}\PY{p}{[}\PY{n}{idx}\PY{p}{]}
        \PY{n+nb}{print}\PY{p}{(}\PY{l+s+s1}{\PYZsq{}}\PY{l+s+s1}{Eigenvalues Sorted:}\PY{l+s+s1}{\PYZsq{}}\PY{p}{)}
        \PY{n+nb}{print}\PY{p}{(}\PY{n}{eigenvalues}\PY{p}{)}
        \PY{n}{eigenvectors} \PY{o}{=} \PY{n}{eigenvectors}\PY{p}{[}\PY{p}{:}\PY{p}{,}\PY{n}{idx}\PY{p}{]}
        \PY{n+nb}{print}\PY{p}{(}\PY{l+s+s1}{\PYZsq{}}\PY{l+s+s1}{Eigenvectors Sorted:}\PY{l+s+s1}{\PYZsq{}}\PY{p}{)}
        \PY{n+nb}{print}\PY{p}{(}\PY{n}{eigenvectors}\PY{p}{)}
        
        \PY{c+c1}{\PYZsh{} COMPUTE U, S, V}
        \PY{n}{U} \PY{o}{=} \PY{n}{eigenvectors}
        \PY{n}{S} \PY{o}{=} \PY{n}{eigenvalues} \PY{o}{*} \PY{n}{np}\PY{o}{.}\PY{n}{eye}\PY{p}{(}\PY{n}{column\PYZus{}num}\PY{p}{)}
        \PY{n}{V} \PY{o}{=} \PY{n}{U}\PY{o}{.}\PY{n}{T}
        
        \PY{n+nb}{print}\PY{p}{(}\PY{l+s+s2}{\PYZdq{}}\PY{l+s+se}{\PYZbs{}n}\PY{l+s+s2}{U:}\PY{l+s+s2}{\PYZdq{}}\PY{p}{)}
        \PY{n+nb}{print}\PY{p}{(}\PY{n}{U}\PY{p}{)}
        \PY{n+nb}{print}\PY{p}{(}\PY{l+s+s2}{\PYZdq{}}\PY{l+s+s2}{S:}\PY{l+s+s2}{\PYZdq{}}\PY{p}{)}
        \PY{n+nb}{print}\PY{p}{(}\PY{n}{S}\PY{p}{)}
        \PY{n+nb}{print}\PY{p}{(}\PY{l+s+s2}{\PYZdq{}}\PY{l+s+s2}{V:}\PY{l+s+s2}{\PYZdq{}}\PY{p}{)}
        \PY{n+nb}{print}\PY{p}{(}\PY{n}{V}\PY{p}{)}
        
        \PY{c+c1}{\PYZsh{} CHECK IF ORIGINAL COVARIANCE MATRIX IS RETURNED}
        \PY{n}{D} \PY{o}{=} \PY{n}{np}\PY{o}{.}\PY{n}{dot}\PY{p}{(}\PY{n}{U}\PY{p}{,} \PY{n}{np}\PY{o}{.}\PY{n}{dot}\PY{p}{(}\PY{n}{S}\PY{p}{,}  \PY{n}{V}\PY{p}{)}\PY{p}{)}
        \PY{n}{np}\PY{o}{.}\PY{n}{allclose}\PY{p}{(}\PY{n}{C}\PY{p}{,}\PY{n}{D}\PY{p}{)}
\end{Verbatim}


    \begin{Verbatim}[commandchars=\\\{\}]
Eigenvalues Sorted:
[0.08943556 0.0423127  0.01458897 0.01284528 0.00850822 0.00492059
 0.00065696]
Eigenvectors Sorted:
[[-3.41720629e-01 -7.29824958e-01  4.56914809e-01  3.52555665e-01
  -1.29441525e-01  2.62199804e-02 -8.56193962e-03]
 [-9.17492644e-02 -5.28794379e-01 -7.04480258e-01 -3.42514767e-01
  -1.39018617e-01  2.81111220e-01  2.89703065e-04]
 [-1.99698823e-02 -7.25012267e-02  9.24097367e-02 -1.46705018e-02
   8.54762221e-01  4.93627297e-01 -1.06318528e-01]
 [ 9.74209562e-04 -1.16809378e-02  8.81896510e-03  1.64849996e-02
   8.61437517e-02  6.21081678e-02  9.94100049e-01]
 [-1.47115119e-01 -4.80496808e-02  4.70463079e-01 -8.66343630e-01
  -4.86492931e-02 -3.90505852e-02  1.64278490e-02]
 [-6.89914858e-01  7.21275410e-02 -2.54072739e-01 -1.83906883e-02
   3.50569568e-01 -5.75266798e-01  9.64476070e-03]
 [-6.13827312e-01  4.18124394e-01  2.07514962e-02  8.37356323e-02
  -3.17196689e-01  5.83358230e-01 -5.01770938e-03]]

U:
[[-3.41720629e-01 -7.29824958e-01  4.56914809e-01  3.52555665e-01
  -1.29441525e-01  2.62199804e-02 -8.56193962e-03]
 [-9.17492644e-02 -5.28794379e-01 -7.04480258e-01 -3.42514767e-01
  -1.39018617e-01  2.81111220e-01  2.89703065e-04]
 [-1.99698823e-02 -7.25012267e-02  9.24097367e-02 -1.46705018e-02
   8.54762221e-01  4.93627297e-01 -1.06318528e-01]
 [ 9.74209562e-04 -1.16809378e-02  8.81896510e-03  1.64849996e-02
   8.61437517e-02  6.21081678e-02  9.94100049e-01]
 [-1.47115119e-01 -4.80496808e-02  4.70463079e-01 -8.66343630e-01
  -4.86492931e-02 -3.90505852e-02  1.64278490e-02]
 [-6.89914858e-01  7.21275410e-02 -2.54072739e-01 -1.83906883e-02
   3.50569568e-01 -5.75266798e-01  9.64476070e-03]
 [-6.13827312e-01  4.18124394e-01  2.07514962e-02  8.37356323e-02
  -3.17196689e-01  5.83358230e-01 -5.01770938e-03]]
S:
[[0.08943556 0.         0.         0.         0.         0.
  0.        ]
 [0.         0.0423127  0.         0.         0.         0.
  0.        ]
 [0.         0.         0.01458897 0.         0.         0.
  0.        ]
 [0.         0.         0.         0.01284528 0.         0.
  0.        ]
 [0.         0.         0.         0.         0.00850822 0.
  0.        ]
 [0.         0.         0.         0.         0.         0.00492059
  0.        ]
 [0.         0.         0.         0.         0.         0.
  0.00065696]]
V:
[[-3.41720629e-01 -9.17492644e-02 -1.99698823e-02  9.74209562e-04
  -1.47115119e-01 -6.89914858e-01 -6.13827312e-01]
 [-7.29824958e-01 -5.28794379e-01 -7.25012267e-02 -1.16809378e-02
  -4.80496808e-02  7.21275410e-02  4.18124394e-01]
 [ 4.56914809e-01 -7.04480258e-01  9.24097367e-02  8.81896510e-03
   4.70463079e-01 -2.54072739e-01  2.07514962e-02]
 [ 3.52555665e-01 -3.42514767e-01 -1.46705018e-02  1.64849996e-02
  -8.66343630e-01 -1.83906883e-02  8.37356323e-02]
 [-1.29441525e-01 -1.39018617e-01  8.54762221e-01  8.61437517e-02
  -4.86492931e-02  3.50569568e-01 -3.17196689e-01]
 [ 2.62199804e-02  2.81111220e-01  4.93627297e-01  6.21081678e-02
  -3.90505852e-02 -5.75266798e-01  5.83358230e-01]
 [-8.56193962e-03  2.89703065e-04 -1.06318528e-01  9.94100049e-01
   1.64278490e-02  9.64476070e-03 -5.01770938e-03]]

    \end{Verbatim}

\begin{Verbatim}[commandchars=\\\{\}]
{\color{outcolor}Out[{\color{outcolor}8}]:} True
\end{Verbatim}
            
    \subsubsection{Problem 2d (10 points)}\label{problem-2d-10-points}

Compute the projection of the raw data onto the appropriate two
eigenvectors. Consider which columns should be projected and the
normalizations.

    \begin{Verbatim}[commandchars=\\\{\}]
{\color{incolor}In [{\color{incolor}10}]:} \PY{c+c1}{\PYZsh{} The data was centered so there is not need to normalize the eigenvectors}
         \PY{c+c1}{\PYZsh{} The U matrix is sorted from highest eigenvalue to lowest in previous section so no need to re\PYZhy{}arrange here.}
         
         \PY{n}{Comp\PYZus{}num} \PY{o}{=} \PY{l+m+mi}{2}
         
         \PY{c+c1}{\PYZsh{} PROJECT DATA ON TOP TWO EIGENVECTORS    }
         \PY{n}{ecoli\PYZus{}PCA} \PY{o}{=}\PY{n}{np}\PY{o}{.}\PY{n}{dot}\PY{p}{(}\PY{n}{ecoli0}\PY{p}{,}\PY{n}{U}\PY{p}{[}\PY{p}{:}\PY{p}{,}\PY{l+m+mi}{0}\PY{p}{:}\PY{n}{Comp\PYZus{}num}\PY{p}{]}\PY{p}{)}
         \PY{n+nb}{print}\PY{p}{(}\PY{n}{ecoli\PYZus{}PCA}\PY{p}{)}
\end{Verbatim}


    \begin{Verbatim}[commandchars=\\\{\}]
[[ 0.28560071  0.03527368]
 [ 0.29083817  0.330159  ]
 [ 0.10467597 -0.0152477 ]
 [ 0.08794301 -0.12221767]
 [ 0.3662676   0.21036611]
 [ 0.08023001 -0.08327277]
 [ 0.38555357  0.18739006]
 [ 0.33190109  0.23519751]
 [ 0.0643298   0.28387054]
 [ 0.37151499  0.00296018]
 [ 0.27966348  0.12284697]
 [ 0.4529588   0.08559031]
 [ 0.33173181  0.09719732]
 [ 0.21901728 -0.07979446]
 [ 0.4749507   0.12710027]
 [ 0.12989914  0.24257591]
 [ 0.34865965  0.06878927]
 [-0.02836356  0.19918564]
 [ 0.34335381  0.19045807]
 [ 0.3513704  -0.06094186]
 [ 0.55679426 -0.12278632]
 [ 0.58759337  0.08869104]
 [ 0.15979858 -0.06365125]
 [ 0.52580325  0.18313219]
 [ 0.42007625  0.02949799]
 [ 0.11604049  0.22828575]
 [ 0.26570136  0.19234895]
 [ 0.27790173  0.15703056]
 [ 0.17204237  0.08376635]
 [ 0.22504571  0.06064461]
 [ 0.2613917  -0.03012105]
 [ 0.22853447  0.17780982]
 [ 0.11665082 -0.06620763]
 [ 0.20581659  0.15325011]
 [ 0.46135353  0.17768333]
 [ 0.16504526  0.06125132]
 [ 0.29438979  0.08667428]
 [ 0.13791083 -0.15288892]
 [ 0.31664858  0.11611992]
 [ 0.36998975  0.00297334]
 [ 0.23945791  0.10021823]
 [ 0.13555289  0.20099448]
 [ 0.13612438  0.05619063]
 [ 0.25287127  0.1795886 ]
 [ 0.12896329 -0.10004694]
 [ 0.18661777  0.30221326]
 [-0.07481577  0.17158965]
 [ 0.26887141  0.04867432]
 [ 0.41227242  0.01373134]
 [ 0.27870455  0.25006429]
 [ 0.19577803  0.02512113]
 [ 0.62458105  0.09082702]
 [ 0.2251905   0.26639531]
 [ 0.0283988   0.20207408]
 [ 0.43637704 -0.07237731]
 [ 0.1881318  -0.04718646]
 [ 0.15814802  0.03893181]
 [ 0.22197303  0.11027592]
 [ 0.22515069  0.09775956]
 [ 0.18610835  0.08039364]
 [ 0.01833289  0.02718512]
 [ 0.2449825  -0.00100039]
 [ 0.05547609  0.16657255]
 [ 0.40532813  0.26103644]
 [ 0.18859236  0.0922141 ]
 [ 0.02063356  0.26520527]
 [ 0.27728074  0.21109576]
 [-0.00505219 -0.03192357]
 [ 0.35946212  0.17999838]
 [ 0.12582531  0.10287482]
 [ 0.39241511  0.23028946]
 [ 0.2758057  -0.03462157]
 [ 0.18376345  0.06900056]
 [ 0.3965304  -0.05224877]
 [ 0.34012643  0.16825698]
 [ 0.3004363   0.09054126]
 [ 0.28637403 -0.0201646 ]
 [ 0.06666642  0.07691664]
 [ 0.28975521  0.09382565]
 [ 0.19059505  0.17771886]
 [ 0.22870848  0.20280805]
 [ 0.23115245  0.05505032]
 [ 0.37253466  0.08836062]
 [ 0.22493802  0.03643629]
 [ 0.06900602  0.08403208]
 [ 0.00526035  0.11882288]
 [ 0.01186153  0.01671049]
 [ 0.16378088 -0.01888677]
 [ 0.31769947 -0.07932744]
 [ 0.2598017  -0.02311136]
 [ 0.19601094  0.10233959]
 [ 0.36078878  0.01005797]
 [ 0.3703031  -0.01930099]
 [ 0.40438304  0.0842357 ]
 [ 0.31816695  0.10404268]
 [ 0.11309767 -0.29697019]
 [ 0.24625392  0.14830125]
 [ 0.24795757 -0.15165885]
 [ 0.19271491 -0.22830605]
 [ 0.17510294 -0.04004246]
 [ 0.29590036  0.05474484]
 [ 0.21694535  0.10696672]
 [ 0.42704586  0.09734034]
 [ 0.40639913  0.02888906]
 [ 0.32083787  0.22517302]
 [ 0.2705601   0.12871023]
 [ 0.18206702  0.00517858]
 [ 0.08795519  0.16047871]
 [ 0.21096762 -0.01016018]
 [ 0.38233194  0.01046059]
 [ 0.55039742  0.06056855]
 [ 0.03732716  0.22126026]
 [ 0.35127694  0.08669213]
 [ 0.28234862  0.1368443 ]
 [ 0.38070224  0.02287165]
 [ 0.34196091 -0.04324615]
 [ 0.228901    0.12387432]
 [ 0.38906584 -0.12848415]
 [ 0.15912193  0.17892797]
 [ 0.14743047  0.26511122]
 [ 0.29721461  0.26950286]
 [ 0.38335644  0.15273556]
 [ 0.32424186  0.23843235]
 [ 0.40055844  0.38065783]
 [ 0.11733475 -0.04039222]
 [ 0.17711785  0.08954889]
 [ 0.39594958  0.16286558]
 [ 0.19021922  0.06335507]
 [ 0.36291074  0.21463995]
 [ 0.15629441  0.11125454]
 [ 0.17703748  0.45295886]
 [ 0.22632084  0.14699194]
 [ 0.37313814  0.08817971]
 [ 0.2856448   0.20294986]
 [ 0.22693711  0.2059287 ]
 [ 0.23048762  0.09370682]
 [ 0.43439891  0.11264946]
 [ 0.35745222  0.16438102]
 [ 0.33289165  0.02817806]
 [ 0.3348251   0.23294501]
 [-0.02565911  0.1663488 ]
 [ 0.23110945  0.17976915]
 [ 0.35549193  0.21357201]
 [-0.06793977  0.24065691]
 [ 0.02540177  0.05242165]
 [-0.47232529  0.08703261]
 [-0.42041     0.39506455]
 [-0.3815037   0.27018134]
 [-0.32273309  0.23651334]
 [-0.46867564  0.15394197]
 [-0.27310884  0.38871233]
 [-0.30458056  0.14896224]
 [-0.17620367  0.25058282]
 [-0.40166486  0.12292555]
 [-0.0557508   0.23731631]
 [-0.41707814  0.23230157]
 [-0.61054576  0.11119793]
 [-0.39964732 -0.09867178]
 [-0.38032288  0.18314769]
 [-0.48627159  0.16775779]
 [-0.49530852  0.2507292 ]
 [-0.53400861  0.13171157]
 [-0.30584045 -0.04344151]
 [-0.30415289  0.11257235]
 [-0.29943612  0.24742747]
 [-0.38630386 -0.02587031]
 [-0.44506369  0.09253242]
 [-0.35804479  0.11463491]
 [-0.51982097  0.07785112]
 [-0.38372697  0.19997927]
 [-0.39449064  0.30745358]
 [-0.58513294  0.04592758]
 [-0.43123346  0.34467284]
 [-0.32094354  0.21557283]
 [-0.378491    0.00558454]
 [-0.22453518  0.15188023]
 [-0.27897578 -0.05723657]
 [-0.29604724 -0.03172065]
 [-0.40893263 -0.06060438]
 [-0.49245752  0.30069979]
 [-0.46023159  0.02858249]
 [-0.51863736  0.06926185]
 [-0.4205751   0.03359005]
 [ 0.22873538  0.23600789]
 [-0.34372453 -0.19705863]
 [-0.28728342  0.38783347]
 [-0.45041492 -0.18568813]
 [-0.4075664  -0.01004846]
 [-0.1678591   0.42416141]
 [-0.17457982 -0.07087104]
 [-0.14391879  0.28628168]
 [-0.32762264 -0.08053313]
 [-0.31863534  0.19534494]
 [-0.32096105  0.16720861]
 [-0.46674853  0.18092729]
 [-0.44672366  0.05690709]
 [-0.19138966  0.12323758]
 [-0.21112852  0.02901131]
 [-0.43929185  0.11219577]
 [-0.29373785  0.25731919]
 [-0.32723359  0.02786958]
 [-0.21671251  0.22453384]
 [-0.26204259  0.24859382]
 [-0.25775163  0.21568039]
 [-0.32577707  0.32639461]
 [-0.53461631  0.02389202]
 [-0.31008004  0.12748742]
 [-0.14047724  0.37320162]
 [-0.50735284  0.42584825]
 [-0.43738063 -0.01035674]
 [-0.42222923  0.05742981]
 [-0.38525487  0.09977841]
 [ 0.01265413 -0.10452855]
 [ 0.07286098  0.01860053]
 [ 0.11184531  0.35528962]
 [ 0.21186161  0.19383142]
 [ 0.10180869  0.08683856]
 [-0.05747366 -0.09273279]
 [-0.34455879 -0.14903014]
 [-0.73963414  0.07166205]
 [-0.04820124 -0.33273675]
 [-0.40648747  0.04978136]
 [ 0.05855067 -0.33213739]
 [-0.49929013  0.04054778]
 [-0.45390169  0.01654853]
 [-0.53052548 -0.01012169]
 [-0.49009677 -0.09618115]
 [-0.40276666  0.025512  ]
 [-0.42361659 -0.02770693]
 [-0.45437828  0.0241945 ]
 [-0.37910929 -0.00438115]
 [-0.37000461  0.1708445 ]
 [-0.2588356   0.0293947 ]
 [-0.34373534  0.00794068]
 [-0.35921303  0.04652237]
 [-0.40616255 -0.17984001]
 [-0.46282688 -0.12342605]
 [-0.43852679 -0.10214397]
 [-0.31614256 -0.10186661]
 [-0.36773735  0.01099519]
 [-0.40427554  0.09458247]
 [-0.43550641  0.10269954]
 [-0.64047398 -0.01223116]
 [-0.6072157  -0.05861517]
 [-0.49879848 -0.12874011]
 [-0.33475406  0.16876817]
 [-0.21605044 -0.08421317]
 [-0.57571397  0.0105166 ]
 [-0.24207489 -0.00778687]
 [-0.49509968 -0.09326866]
 [-0.41093323  0.09207756]
 [-0.31779603  0.0253273 ]
 [-0.43730234 -0.11358708]
 [-0.56887236 -0.16610282]
 [-0.37702255  0.01897167]
 [-0.56287737 -0.13379916]
 [-0.32335314 -0.1447148 ]
 [ 0.16336102 -0.03700957]
 [-0.35169929 -0.0770857 ]
 [ 0.0374417  -0.40912583]
 [ 0.11350214 -0.29975106]
 [ 0.02695423 -0.43931122]
 [ 0.15992954 -0.29501925]
 [-0.07210282 -0.36557991]
 [-0.02051332 -0.4743981 ]
 [-0.08875028 -0.44570022]
 [-0.08307043 -0.41309171]
 [ 0.00915109 -0.3743173 ]
 [ 0.04305255 -0.47740786]
 [ 0.04058347 -0.3836835 ]
 [ 0.05083137 -0.18166656]
 [ 0.00133115 -0.18936227]
 [ 0.08909816 -0.2466405 ]
 [ 0.18120883 -0.26012916]
 [-0.18221007 -0.2450057 ]
 [ 0.14128623 -0.32713834]
 [ 0.08365929 -0.19783404]
 [ 0.01701157 -0.33526193]
 [ 0.08477446 -0.30071741]
 [ 0.1835235  -0.46623496]
 [-0.02519471 -0.20403302]
 [-0.02045377 -0.24662308]
 [ 0.05719648 -0.33100722]
 [ 0.012207   -0.25664472]
 [-0.01069591 -0.22039497]
 [-0.05780273 -0.2335581 ]
 [ 0.08415265 -0.3631955 ]
 [-0.0060356  -0.39720509]
 [-0.03841681 -0.14174703]
 [ 0.084145   -0.45143735]
 [ 0.06857974 -0.30131124]
 [ 0.08050425 -0.40224993]
 [ 0.09398324 -0.24098621]
 [-0.06870081 -0.35424354]
 [ 0.10934398 -0.16535533]
 [-0.03610165 -0.30133026]
 [ 0.07730555 -0.2711938 ]
 [ 0.15918513 -0.38500197]
 [ 0.17921619 -0.35213511]
 [-0.21324142 -0.25762672]
 [ 0.29292033  0.12333481]
 [-0.06496879 -0.22909376]
 [-0.00291865 -0.24114075]
 [-0.04746685 -0.38096075]
 [ 0.05793017 -0.34773807]
 [ 0.20647878 -0.4008436 ]
 [ 0.05840106 -0.26265205]
 [-0.00689613 -0.40991812]
 [ 0.01389074 -0.31256163]
 [ 0.03625891 -0.31985453]
 [-0.07815722 -0.18641177]
 [-0.00371419 -0.42339406]
 [ 0.09175669 -0.23467383]
 [ 0.18467557 -0.33431481]
 [-0.52496176 -0.08101752]
 [ 0.01274853 -0.38940058]
 [ 0.21104953 -0.31454606]
 [ 0.19421481 -0.35182051]
 [ 0.03166292 -0.28492959]
 [-0.0378903  -0.179018  ]
 [ 0.09185903 -0.29361228]
 [ 0.09700778 -0.30922729]
 [-0.07964565 -0.21691282]
 [ 0.04257996 -0.36113851]
 [ 0.05585319 -0.29159337]
 [ 0.16047059  0.1961778 ]
 [ 0.09286279 -0.31078603]
 [ 0.19887035 -0.15329584]
 [ 0.08738149 -0.32219706]
 [-0.02848041 -0.26202269]
 [-0.01691867  0.02818486]
 [-0.08423256 -0.27480267]
 [ 0.13902577 -0.27411596]
 [ 0.11190373 -0.18710274]
 [ 0.10620378 -0.17885061]
 [-0.10876379 -0.28112947]]

    \end{Verbatim}

    \subsubsection{Problem 2e (10 points)}\label{problem-2e-10-points}

Plot the projected points such that the 8 different classes can be
visually identified. Be sure to label the classes and axes. Commont on
the quality of the separation of the different classes using PCA.

    \begin{Verbatim}[commandchars=\\\{\}]
{\color{incolor}In [{\color{incolor}13}]:} \PY{k+kn}{import} \PY{n+nn}{seaborn} \PY{k}{as} \PY{n+nn}{sns}
         \PY{k+kn}{import} \PY{n+nn}{matplotlib}\PY{n+nn}{.}\PY{n+nn}{pyplot} \PY{k}{as} \PY{n+nn}{plt}
         \PY{k+kn}{from} \PY{n+nn}{matplotlib} \PY{k}{import} \PY{n}{rcParams}
         \PY{k+kn}{import} \PY{n+nn}{matplotlib}\PY{n+nn}{.}\PY{n+nn}{cm} \PY{k}{as} \PY{n+nn}{cm}
         \PY{k+kn}{import} \PY{n+nn}{matplotlib} \PY{k}{as} \PY{n+nn}{mpl}
         \PY{k+kn}{import} \PY{n+nn}{matplotlib}\PY{n+nn}{.}\PY{n+nn}{style}
         
         \PY{k+kn}{from} \PY{n+nn}{matplotlib}\PY{n+nn}{.}\PY{n+nn}{colors} \PY{k}{import} \PY{n}{ListedColormap}
         \PY{c+c1}{\PYZsh{} Remember to use inline to get your plots in the notebook}
         \PY{o}{\PYZpc{}}\PY{k}{matplotlib} inline 
         
         \PY{c+c1}{\PYZsh{} FUNCTION TO DRAW THE PLOT}
         \PY{k}{def} \PY{n+nf}{draw\PYZus{}plot}\PY{p}{(}\PY{n}{df\PYZus{}PCA}\PY{p}{,} \PY{n}{df\PYZus{}orig}\PY{p}{,} \PY{n}{title}\PY{p}{)}\PY{p}{:}
             \PY{n}{cdict} \PY{o}{=} \PY{p}{\PYZob{}}\PY{l+s+s1}{\PYZsq{}}\PY{l+s+s1}{cp}\PY{l+s+s1}{\PYZsq{}}\PY{p}{:}\PY{l+s+s1}{\PYZsq{}}\PY{l+s+s1}{red}\PY{l+s+s1}{\PYZsq{}}\PY{p}{,} \PY{l+s+s1}{\PYZsq{}}\PY{l+s+s1}{im}\PY{l+s+s1}{\PYZsq{}}\PY{p}{:}\PY{l+s+s1}{\PYZsq{}}\PY{l+s+s1}{blue}\PY{l+s+s1}{\PYZsq{}}\PY{p}{,} \PY{l+s+s1}{\PYZsq{}}\PY{l+s+s1}{imS}\PY{l+s+s1}{\PYZsq{}}\PY{p}{:}\PY{l+s+s1}{\PYZsq{}}\PY{l+s+s1}{green}\PY{l+s+s1}{\PYZsq{}}\PY{p}{,} \PY{l+s+s1}{\PYZsq{}}\PY{l+s+s1}{imL}\PY{l+s+s1}{\PYZsq{}}\PY{p}{:}\PY{l+s+s1}{\PYZsq{}}\PY{l+s+s1}{yellow}\PY{l+s+s1}{\PYZsq{}}\PY{p}{,} \PY{l+s+s1}{\PYZsq{}}\PY{l+s+s1}{imU}\PY{l+s+s1}{\PYZsq{}}\PY{p}{:}\PY{l+s+s1}{\PYZsq{}}\PY{l+s+s1}{black}\PY{l+s+s1}{\PYZsq{}}\PY{p}{,} \PY{l+s+s1}{\PYZsq{}}\PY{l+s+s1}{om}\PY{l+s+s1}{\PYZsq{}}\PY{p}{:}\PY{l+s+s1}{\PYZsq{}}\PY{l+s+s1}{cyan}\PY{l+s+s1}{\PYZsq{}}\PY{p}{,} \PY{l+s+s1}{\PYZsq{}}\PY{l+s+s1}{omL}\PY{l+s+s1}{\PYZsq{}}\PY{p}{:}\PY{l+s+s1}{\PYZsq{}}\PY{l+s+s1}{brown}\PY{l+s+s1}{\PYZsq{}}\PY{p}{,} \PY{l+s+s1}{\PYZsq{}}\PY{l+s+s1}{pp}\PY{l+s+s1}{\PYZsq{}}\PY{p}{:}\PY{l+s+s1}{\PYZsq{}}\PY{l+s+s1}{purple}\PY{l+s+s1}{\PYZsq{}}\PY{p}{\PYZcb{}}
             \PY{n}{plt}\PY{o}{.}\PY{n}{figure}\PY{p}{(}\PY{n}{figsize}\PY{o}{=}\PY{p}{(}\PY{l+m+mi}{9}\PY{p}{,}\PY{l+m+mi}{9}\PY{p}{)}\PY{p}{)}
             \PY{n}{plt}\PY{o}{.}\PY{n}{suptitle}\PY{p}{(}\PY{n}{title}\PY{p}{)}
             \PY{k}{for} \PY{n}{c} \PY{o+ow}{in} \PY{n}{df\PYZus{}orig}\PY{p}{[}\PY{l+m+mi}{8}\PY{p}{]}\PY{o}{.}\PY{n}{unique}\PY{p}{(}\PY{p}{)}\PY{p}{:}
                 \PY{n}{idx} \PY{o}{=} \PY{n}{np}\PY{o}{.}\PY{n}{where}\PY{p}{(}\PY{n}{df\PYZus{}orig}\PY{p}{[}\PY{l+m+mi}{8}\PY{p}{]}\PY{o}{==} \PY{n}{c}\PY{p}{)}
                 \PY{n}{plt}\PY{o}{.}\PY{n}{scatter}\PY{p}{(}\PY{n}{df\PYZus{}PCA}\PY{o}{.}\PY{n}{T}\PY{p}{[}\PY{l+m+mi}{0}\PY{p}{]}\PY{p}{[}\PY{n}{idx}\PY{p}{]}\PY{p}{,} \PY{n}{df\PYZus{}PCA}\PY{o}{.}\PY{n}{T}\PY{p}{[}\PY{l+m+mi}{1}\PY{p}{]}\PY{p}{[}\PY{n}{idx}\PY{p}{]}\PY{p}{,} \PY{n}{s}\PY{o}{=}\PY{l+m+mi}{200}\PY{p}{,} \PY{n}{color} \PY{o}{=} \PY{n}{cdict}\PY{p}{[}\PY{n}{c}\PY{p}{]}\PY{p}{,} \PY{n}{alpha} \PY{o}{=} \PY{l+m+mf}{0.6}\PY{p}{,} \PY{n}{label}\PY{o}{=} \PY{n}{c}\PY{p}{)}
             \PY{n}{plt}\PY{o}{.}\PY{n}{legend}\PY{p}{(}\PY{p}{)}
             \PY{n}{plt}\PY{o}{.}\PY{n}{xlabel}\PY{p}{(}\PY{l+s+s2}{\PYZdq{}}\PY{l+s+s2}{Principal Component 1}\PY{l+s+s2}{\PYZdq{}}\PY{p}{)}
             \PY{n}{plt}\PY{o}{.}\PY{n}{ylabel}\PY{p}{(}\PY{l+s+s2}{\PYZdq{}}\PY{l+s+s2}{Principal Component 2}\PY{l+s+s2}{\PYZdq{}}\PY{p}{)}
             \PY{n}{plt}\PY{o}{.}\PY{n}{show}\PY{p}{(}\PY{p}{)}    
         
         \PY{n}{draw\PYZus{}plot}\PY{p}{(}\PY{n}{ecoli\PYZus{}PCA}\PY{p}{,} \PY{n}{ecoli}\PY{p}{,} \PY{l+s+s1}{\PYZsq{}}\PY{l+s+s1}{The data points projected on 2 PCA dimensions}\PY{l+s+s1}{\PYZsq{}}\PY{p}{)}
\end{Verbatim}


    \begin{center}
    \adjustimage{max size={0.9\linewidth}{0.9\paperheight}}{output_15_0.png}
    \end{center}
    { \hspace*{\fill} \\}
    
    The two PCA components are able to clearly seperate between 'cp', 'im'
and 'pp'. But these two PCA components are not able to capture the
difference between 'im' and 'imU'. It is also not able to capture the
difference between 'om' and 'pp'.

So we can conclude these 2 PCA dimensions are able to seperate between
some classes very well. But we might need to look at some of the other
dimensions to see if we can find the difference between the classes that
are not very well seperated on these 2 dimensions.

    \subsubsection{Problem 2f (10 points)}\label{problem-2f-10-points}

The PCA that you have just completed takes each data point and projects
it using a weighted sum of features. One could also do the opposite to
map the features as a weighted sum of the data entries. How could this
be done? What is a potential issue? Describe these in a few sentences
(do not code it).

    We could take the transpose of the original data matrix such that each
data entry would become seperate columns(features), and the features
will become rows.Then we can apply the same PCA dimensionality reduction
techniques to map the features into a weighted sum of the data entries.

Since, we have a lot of data entries, the covariance matrix will be very
large. If the covariance matrix is very large, each eigenvector will
likely capture a small percentage of the variance. So, we will have to
include many components to capture enough variance of the data to be
able to do meaningful analysis.

    \subsubsection{Problem 3 MDS (10 points)}\label{problem-3-mds-10-points}

For the same data set, repeat 2e using sklearn's Multidimensional
scaling algorithm.

    \begin{Verbatim}[commandchars=\\\{\}]
{\color{incolor}In [{\color{incolor}14}]:} \PY{k+kn}{from} \PY{n+nn}{sklearn}\PY{n+nn}{.}\PY{n+nn}{manifold} \PY{k}{import} \PY{n}{MDS}
         \PY{n}{ecoli\PYZus{}PCA\PYZus{}MDS} \PY{o}{=} \PY{n}{MDS}\PY{p}{(}\PY{n}{n\PYZus{}components} \PY{o}{=} \PY{l+m+mi}{2}\PY{p}{)}\PY{o}{.}\PY{n}{fit\PYZus{}transform}\PY{p}{(}\PY{n}{ecoli0}\PY{p}{)}
          
         \PY{c+c1}{\PYZsh{}print(ecoli\PYZus{}PCA\PYZus{}MDS)}
         \PY{n}{draw\PYZus{}plot}\PY{p}{(}\PY{n}{ecoli\PYZus{}PCA\PYZus{}MDS}\PY{p}{,} \PY{n}{ecoli}\PY{p}{,} \PY{l+s+s1}{\PYZsq{}}\PY{l+s+s1}{The data points on 2 PCA (MDS) dimensions}\PY{l+s+s1}{\PYZsq{}}\PY{p}{)}
\end{Verbatim}


    \begin{center}
    \adjustimage{max size={0.9\linewidth}{0.9\paperheight}}{output_20_0.png}
    \end{center}
    { \hspace*{\fill} \\}
    
    \subsubsection{Problem 4a t-SNE (5
points)}\label{problem-4a-t-sne-5-points}

Repeat 2e using a t-SNE plot with the default settings.

    \begin{Verbatim}[commandchars=\\\{\}]
{\color{incolor}In [{\color{incolor}15}]:} \PY{k+kn}{from} \PY{n+nn}{sklearn}\PY{n+nn}{.}\PY{n+nn}{manifold} \PY{k}{import} \PY{n}{TSNE}
         
         \PY{n}{ecoli\PYZus{}TSNE} \PY{o}{=} \PY{n}{TSNE}\PY{p}{(}\PY{n}{n\PYZus{}components}\PY{o}{=}\PY{l+m+mi}{3}\PY{p}{,} \PY{n}{perplexity}\PY{o}{=}\PY{l+m+mi}{35}\PY{p}{,} \PY{n}{verbose}\PY{o}{=}\PY{l+m+mi}{0}\PY{p}{)}\PY{o}{.}\PY{n}{fit\PYZus{}transform}\PY{p}{(}\PY{n}{ecoli0}\PY{p}{)}
         \PY{n}{draw\PYZus{}plot}\PY{p}{(}\PY{n}{ecoli\PYZus{}TSNE}\PY{p}{,} \PY{n}{ecoli}\PY{p}{,} \PY{l+s+s1}{\PYZsq{}}\PY{l+s+s1}{The data points on 2 TSNE dimensions}\PY{l+s+s1}{\PYZsq{}}\PY{p}{)}
\end{Verbatim}


    \begin{center}
    \adjustimage{max size={0.9\linewidth}{0.9\paperheight}}{output_22_0.png}
    \end{center}
    { \hspace*{\fill} \\}
    
    \subsubsection{Problem 4b t-SNE perplexity (5
points)}\label{problem-4b-t-sne-perplexity-5-points}

Try out a few t-SNE plots by varying the perplexity. State the best
perplexity for separating the 8 different classes and describe your
rationale in a sentence or two. Report the average calculation time for
the t-SNE projection over a number of iterations.

    \begin{Verbatim}[commandchars=\\\{\}]
{\color{incolor}In [{\color{incolor}16}]:} \PY{k+kn}{import} \PY{n+nn}{time}
         
         \PY{n}{starttime} \PY{o}{=} \PY{n}{time}\PY{o}{.}\PY{n}{time}\PY{p}{(}\PY{p}{)}
         \PY{n}{ecoli\PYZus{}TSNE} \PY{o}{=} \PY{n}{TSNE}\PY{p}{(}\PY{n}{n\PYZus{}components}\PY{o}{=}\PY{l+m+mi}{3}\PY{p}{,} \PY{n}{perplexity}\PY{o}{=}\PY{l+m+mi}{35}\PY{p}{,} \PY{n}{verbose}\PY{o}{=}\PY{l+m+mi}{0}\PY{p}{)}\PY{o}{.}\PY{n}{fit\PYZus{}transform}\PY{p}{(}\PY{n}{ecoli0}\PY{p}{)}
         \PY{n}{draw\PYZus{}plot}\PY{p}{(}\PY{n}{ecoli\PYZus{}TSNE}\PY{p}{,} \PY{n}{ecoli}\PY{p}{,} \PY{l+s+s1}{\PYZsq{}}\PY{l+s+s1}{The data points on 2 TSNE dimensions}\PY{l+s+s1}{\PYZsq{}}\PY{p}{)}
         \PY{n}{endtime} \PY{o}{=} \PY{n}{time}\PY{o}{.}\PY{n}{time}\PY{p}{(}\PY{p}{)}
         \PY{n+nb}{print}\PY{p}{(}\PY{l+s+s2}{\PYZdq{}}\PY{l+s+si}{\PYZob{}:03.9f\PYZcb{}}\PY{l+s+s2}{ seconds}\PY{l+s+s2}{\PYZdq{}}\PY{o}{.}\PY{n}{format}\PY{p}{(}\PY{n}{endtime}\PY{o}{\PYZhy{}}\PY{n}{starttime}\PY{p}{)}\PY{p}{)}
         
         \PY{c+c1}{\PYZsh{} AVERAGE TIME TO COMPUTE TSNE PROJECTION}
         \PY{n}{times} \PY{o}{=} \PY{p}{[}\PY{p}{]}
         \PY{n}{n} \PY{o}{=} \PY{l+m+mi}{0}
         \PY{k}{for} \PY{n}{i} \PY{o+ow}{in} \PY{n+nb}{range}\PY{p}{(}\PY{l+m+mi}{5}\PY{p}{)}\PY{p}{:}
             \PY{n}{perplexity} \PY{o}{=} \PY{p}{(}\PY{l+m+mi}{25}\PY{o}{+}\PY{p}{(}\PY{n}{n}\PY{o}{*}\PY{l+m+mi}{5}\PY{p}{)}\PY{p}{)}
             \PY{n}{starttime} \PY{o}{=} \PY{n}{time}\PY{o}{.}\PY{n}{time}\PY{p}{(}\PY{p}{)}
             \PY{n+nb}{print}\PY{p}{(}\PY{l+s+s1}{\PYZsq{}}\PY{l+s+s1}{Perplexity = }\PY{l+s+si}{\PYZpc{}d}\PY{l+s+s1}{\PYZsq{}}\PY{o}{\PYZpc{}}\PY{p}{(}\PY{n}{perplexity}\PY{p}{)}\PY{p}{)}
             \PY{n}{ecoli\PYZus{}TSNE} \PY{o}{=} \PY{n}{TSNE}\PY{p}{(}\PY{n}{n\PYZus{}components}\PY{o}{=}\PY{l+m+mi}{3}\PY{p}{,} \PY{n}{perplexity}\PY{o}{=} \PY{n}{perplexity}\PY{p}{,} \PY{n}{verbose}\PY{o}{=}\PY{l+m+mi}{0}\PY{p}{)}\PY{o}{.}\PY{n}{fit\PYZus{}transform}\PY{p}{(}\PY{n}{ecoli0}\PY{p}{)}
             \PY{n}{endtime} \PY{o}{=} \PY{n}{time}\PY{o}{.}\PY{n}{time}\PY{p}{(}\PY{p}{)}
             \PY{n}{draw\PYZus{}plot}\PY{p}{(}\PY{n}{ecoli\PYZus{}TSNE}\PY{p}{,} \PY{n}{ecoli}\PY{p}{,} \PY{l+s+s1}{\PYZsq{}}\PY{l+s+s1}{The data points on 2 TSNE dimensions}\PY{l+s+s1}{\PYZsq{}}\PY{p}{)}
             \PY{n}{times}\PY{o}{.}\PY{n}{append}\PY{p}{(}\PY{n}{endtime}\PY{o}{\PYZhy{}}\PY{n}{starttime}\PY{p}{)}
             \PY{n}{n} \PY{o}{=} \PY{n}{n} \PY{o}{+} \PY{l+m+mi}{1}
             
         \PY{n+nb}{print}\PY{p}{(}\PY{l+s+s1}{\PYZsq{}}\PY{l+s+s1}{Average Time for TSNE projection: }\PY{l+s+si}{\PYZob{}:03.9f\PYZcb{}}\PY{l+s+s1}{ seconds}\PY{l+s+s1}{\PYZsq{}}\PY{o}{.}\PY{n}{format}\PY{p}{(}\PY{n}{np}\PY{o}{.}\PY{n}{mean}\PY{p}{(}\PY{n}{times}\PY{p}{)}\PY{p}{)}\PY{p}{)}
\end{Verbatim}


    \begin{center}
    \adjustimage{max size={0.9\linewidth}{0.9\paperheight}}{output_24_0.png}
    \end{center}
    { \hspace*{\fill} \\}
    
    \begin{Verbatim}[commandchars=\\\{\}]
12.971605778 seconds
Perplexity = 25

    \end{Verbatim}

    \begin{center}
    \adjustimage{max size={0.9\linewidth}{0.9\paperheight}}{output_24_2.png}
    \end{center}
    { \hspace*{\fill} \\}
    
    \begin{Verbatim}[commandchars=\\\{\}]
Perplexity = 30

    \end{Verbatim}

    \begin{center}
    \adjustimage{max size={0.9\linewidth}{0.9\paperheight}}{output_24_4.png}
    \end{center}
    { \hspace*{\fill} \\}
    
    \begin{Verbatim}[commandchars=\\\{\}]
Perplexity = 35

    \end{Verbatim}

    \begin{center}
    \adjustimage{max size={0.9\linewidth}{0.9\paperheight}}{output_24_6.png}
    \end{center}
    { \hspace*{\fill} \\}
    
    \begin{Verbatim}[commandchars=\\\{\}]
Perplexity = 40

    \end{Verbatim}

    \begin{center}
    \adjustimage{max size={0.9\linewidth}{0.9\paperheight}}{output_24_8.png}
    \end{center}
    { \hspace*{\fill} \\}
    
    \begin{Verbatim}[commandchars=\\\{\}]
Perplexity = 45

    \end{Verbatim}

    \begin{center}
    \adjustimage{max size={0.9\linewidth}{0.9\paperheight}}{output_24_10.png}
    \end{center}
    { \hspace*{\fill} \\}
    
    \begin{Verbatim}[commandchars=\\\{\}]
Average Time for TSNE projection: 13.229911995 seconds

    \end{Verbatim}

    Looking at the distribution of the different Perplexities, it looks like
40 is the best value for perplexity because it groups the different
classes with the least amount of overlap for this value.

The average Time for TSNE projections is about 13.22 seconds.

    \subsubsection{Problem 4c t-SNE randomization (10
points)}\label{problem-4c-t-sne-randomization-10-points}

The S of t-SNE means stochastic or random, usually as a function of
time. Explore whether you can reproduce the result in 4b through a
second projection and plot.

    \begin{Verbatim}[commandchars=\\\{\}]
{\color{incolor}In [{\color{incolor}17}]:} \PY{n}{ecoli\PYZus{}TSNE\PYZus{}second} \PY{o}{=} \PY{n}{TSNE}\PY{p}{(}\PY{n}{n\PYZus{}components}\PY{o}{=}\PY{l+m+mi}{3}\PY{p}{,} \PY{n}{perplexity}\PY{o}{=}\PY{l+m+mi}{35}\PY{p}{,} \PY{n}{verbose}\PY{o}{=}\PY{l+m+mi}{0}\PY{p}{)}\PY{o}{.}\PY{n}{fit\PYZus{}transform}\PY{p}{(}\PY{n}{ecoli0}\PY{p}{)}
         \PY{n+nb}{print}\PY{p}{(}\PY{n}{np}\PY{o}{.}\PY{n}{allclose}\PY{p}{(}\PY{n}{ecoli\PYZus{}TSNE}\PY{p}{,}\PY{n}{ecoli\PYZus{}TSNE\PYZus{}second}\PY{p}{)}\PY{p}{)}
         \PY{n}{draw\PYZus{}plot}\PY{p}{(}\PY{n}{ecoli\PYZus{}TSNE\PYZus{}second}\PY{p}{,} \PY{n}{ecoli}\PY{p}{,} \PY{l+s+s1}{\PYZsq{}}\PY{l+s+s1}{The data points on 2 TSNE dimensions}\PY{l+s+s1}{\PYZsq{}}\PY{p}{)}
\end{Verbatim}


    \begin{Verbatim}[commandchars=\\\{\}]
False

    \end{Verbatim}

    \begin{center}
    \adjustimage{max size={0.9\linewidth}{0.9\paperheight}}{output_27_1.png}
    \end{center}
    { \hspace*{\fill} \\}
    
    Running np.allclose returns false on the second TSNE showing there is
variation between the two calculations of TSNE in 4b and 4c.\\
Looking at the plot it is also clear that there is some variation
between the two runs which shows unlike PCA, TSNE has randomness in it.

    \subsubsection{Problem 4d t-SNE Barnes-Hut (5
points)}\label{problem-4d-t-sne-barnes-hut-5-points}

The default t-SNE method of 4b uses the Barnes-Hut approximation.
Keeping the other parameters the same as 4b, plot the t-SNE result using
the exact method. Which method do you prefer? Compare the average
calculation time for the exact method over a number of iterations.

    \begin{Verbatim}[commandchars=\\\{\}]
{\color{incolor}In [{\color{incolor}18}]:} \PY{n}{starttime} \PY{o}{=} \PY{n}{time}\PY{o}{.}\PY{n}{time}\PY{p}{(}\PY{p}{)}
         \PY{n}{ecoli\PYZus{}TSNE} \PY{o}{=} \PY{n}{TSNE}\PY{p}{(}\PY{n}{n\PYZus{}components}\PY{o}{=}\PY{l+m+mi}{3}\PY{p}{,} \PY{n}{method}\PY{o}{=}\PY{l+s+s1}{\PYZsq{}}\PY{l+s+s1}{exact}\PY{l+s+s1}{\PYZsq{}}\PY{p}{,} \PY{n}{perplexity}\PY{o}{=}\PY{l+m+mi}{35}\PY{p}{,} \PY{n}{verbose}\PY{o}{=}\PY{l+m+mi}{0}\PY{p}{)}\PY{o}{.}\PY{n}{fit\PYZus{}transform}\PY{p}{(}\PY{n}{ecoli0}\PY{p}{)}
         \PY{n}{draw\PYZus{}plot}\PY{p}{(}\PY{n}{ecoli\PYZus{}TSNE}\PY{p}{,} \PY{n}{ecoli}\PY{p}{,} \PY{l+s+s1}{\PYZsq{}}\PY{l+s+s1}{The data points on 2 TSNE (exact) dimensions}\PY{l+s+s1}{\PYZsq{}}\PY{p}{)}
         \PY{n}{endtime} \PY{o}{=} \PY{n}{time}\PY{o}{.}\PY{n}{time}\PY{p}{(}\PY{p}{)}
         \PY{n+nb}{print}\PY{p}{(}\PY{l+s+s2}{\PYZdq{}}\PY{l+s+si}{\PYZob{}:03.9f\PYZcb{}}\PY{l+s+s2}{ seconds}\PY{l+s+s2}{\PYZdq{}}\PY{o}{.}\PY{n}{format}\PY{p}{(}\PY{n}{endtime}\PY{o}{\PYZhy{}}\PY{n}{starttime}\PY{p}{)}\PY{p}{)}
         
         \PY{c+c1}{\PYZsh{} AVERAGE TIME TO COMPUTE TSNE PROJECTION}
         \PY{n}{times} \PY{o}{=} \PY{p}{[}\PY{p}{]}
         \PY{k}{for} \PY{n}{i} \PY{o+ow}{in} \PY{n+nb}{range}\PY{p}{(}\PY{l+m+mi}{2}\PY{p}{)}\PY{p}{:}
             \PY{n}{starttime} \PY{o}{=} \PY{n}{time}\PY{o}{.}\PY{n}{time}\PY{p}{(}\PY{p}{)}
             \PY{n}{ecoli\PYZus{}TSNE} \PY{o}{=} \PY{n}{TSNE}\PY{p}{(}\PY{n}{n\PYZus{}components}\PY{o}{=}\PY{l+m+mi}{3}\PY{p}{,} \PY{n}{method}\PY{o}{=}\PY{l+s+s1}{\PYZsq{}}\PY{l+s+s1}{exact}\PY{l+s+s1}{\PYZsq{}}\PY{p}{,} \PY{n}{perplexity}\PY{o}{=}\PY{l+m+mi}{35}\PY{p}{,} \PY{n}{verbose}\PY{o}{=}\PY{l+m+mi}{0}\PY{p}{)}\PY{o}{.}\PY{n}{fit\PYZus{}transform}\PY{p}{(}\PY{n}{ecoli0}\PY{p}{)}
             \PY{n}{endtime} \PY{o}{=} \PY{n}{time}\PY{o}{.}\PY{n}{time}\PY{p}{(}\PY{p}{)}
             \PY{n}{times}\PY{o}{.}\PY{n}{append}\PY{p}{(}\PY{n}{endtime}\PY{o}{\PYZhy{}}\PY{n}{starttime}\PY{p}{)}
         \PY{n+nb}{print}\PY{p}{(}\PY{l+s+s1}{\PYZsq{}}\PY{l+s+s1}{Average Time for TSNE(exact) projection: }\PY{l+s+si}{\PYZob{}:03.9f\PYZcb{}}\PY{l+s+s1}{ seconds}\PY{l+s+s1}{\PYZsq{}}\PY{o}{.}\PY{n}{format}\PY{p}{(}\PY{n}{np}\PY{o}{.}\PY{n}{mean}\PY{p}{(}\PY{n}{times}\PY{p}{)}\PY{p}{)}\PY{p}{)}
\end{Verbatim}


    \begin{center}
    \adjustimage{max size={0.9\linewidth}{0.9\paperheight}}{output_30_0.png}
    \end{center}
    { \hspace*{\fill} \\}
    
    \begin{Verbatim}[commandchars=\\\{\}]
13.064965248 seconds
Average Time for TSNE(exact) projection: 13.893085599 seconds

    \end{Verbatim}

    The data classes were better seperated using the Barnes-Hut
approximation.\\
The Average time for computations is also around \textasciitilde{}13
seconds using both methods.\\
So, I would prefer the default mehtod over the 'exact' method for this
computation.

    \subsubsection{How many hours did this homework
take?}\label{how-many-hours-did-this-homework-take}

This will not affect your grade. We will be monitoring time spent on
homework to be sure that we are not over-burdening students.

    \begin{Verbatim}[commandchars=\\\{\}]
{\color{incolor}In [{\color{incolor} }]:} \PY{l+m+mi}{16}
\end{Verbatim}


    \subsubsection{Last step (5 points)}\label{last-step-5-points}

Save this notebook as LastnameFirstnameHW1.ipynb such as
MuskElonHW1.ipynb. Create a pdf of this notebook named similarly. Submit
both the python notebook and the pdf version to the Canvas dropbox. We
require both versions.


    % Add a bibliography block to the postdoc
    
    
    
    \end{document}
